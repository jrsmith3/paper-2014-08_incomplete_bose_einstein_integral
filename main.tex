% This file was derived from a template created by Joshua Ryan Smith 
% (joshua.r.smith@gmail.com). The template can be found in the git repo at:
% http://github.com/jrsmith3/latex_template

\documentclass[letterpaper,12pt]{article}

% Document graphics and formatting
% ================================
\usepackage{graphicx}
\graphicspath{{../pics/}}
\usepackage{showkeys}
% \usepackage{endfloat}
% \usepackage{url}
\usepackage{amsmath}
\usepackage[version=3]{mhchem}

% Document metadata
% =================
\newcommand{\Title}{Analytic solution to the incomplete Bose-Einstein integral}
\newcommand{\AuthorName}{Joshua Ryan Smith}
\newcommand{\AuthorEmail}{joshua.smith133.ctr@mail.mil}

\usepackage[pdftex,colorlinks=true,hidelinks]{hyperref}
\hypersetup{
pdftitle={\Title},
pdfauthor={\AuthorName (\AuthorEmail)},
pdfsubject={},
pdfkeywords={},
pdfcreator={pdfTeX}
}

% Label index
% ===========
% eq:
% --
% 00,01,02,03,04,05,06,07,08,09,
% 10,11,12,13,14,15,16,17,18,19,
% 20,21,22,23,24,
% 
% fig
% ---
%
% sec
% ---
%
% tab
% ---

\title{\Title}
\author{\AuthorName}

\begin{document}

\maketitle


\begin{abstract}

\end{abstract}


\section{Introduction}
The incomplete Bose-Einstein integral appears when calculating the flux of above-bandgap photons thermally emitted by an idealized semiconductor. This integral appears when determining the detailed-balance limit of a solar cell \cite{10.1063/1.1736034}, and also when calculating the conduction band population of a material experiencing photon-enhanced thermoelectron emission \cite{10.1038/nmat2814}. Despite the importance of this calculation, everyone seems to leave the details up to the reader. Herein I calculate an analytical solution to the incomplete Bose-Einstein integral in terms of polylogarithm functions.

\section{Calculation}
Shockly and Quiesser \cite{10.1063/1.1736034} give an expression for $Q(\nu_{g}, T)$; the number of photons of frequency greater than $\nu_{g}$ leaving a material per unit area per unit time for blackbody radiation of temperature $T$. This expression is represented in Eq. \ref{eq:00}

\begin{equation} \label{eq:00}
Q(\nu_{g}, T) = \frac{2\pi}{c^{2}} \int_{\nu_{g}}^{\infty} \frac{\nu^{2}}{\exp\left( \frac{h \nu}{kT} \right) - 1} d\nu
\end{equation}

\noindent where $c$ is the speed of light in vacuum, $h$ is Planck's constant, $k$ is Boltzmann's constant, and $\nu$ is the frequency of a photon.

The variable an be changed to simplify the integral as shown in Eq. \ref{eq:01}

\begin{equation} \label{eq:01}
Q(\nu_{g}, T) = \frac{2\pi (kT)^{3}}{h^{3} c^{2}} \int_{x_{g}}^{\infty} \frac{x^{2}}{\exp(x) - 1} dx
\end{equation}

\noindent where $x$ is defined as

\begin{equation} \label{eq:02}
x_{g}kT = h \nu_{g} = q V_{g}
\end{equation}

\noindent and $q$ is the magnitude of the electron charge and $V_{g}$ is the bandgap of the semiconductor.

Note the series

\begin{equation} \label{eq:03}
\sum_{k = 0}^{\infty} r^{k} = \frac{1}{1-r}, |r| < 1
\end{equation}

\noindent Multiplying through by $r$,

\begin{equation} \label{eq:04}
\sum_{k = 1}^{\infty} r^{k} = \frac{r}{1-r}, |r| < 1
\end{equation}

In the present case,

\begin{equation} \label{eq:05}
\frac{1}{\exp(x) - 1} = \frac{\exp(-x)}{1 - \exp(-x)}
\end{equation}

\noindent and $x > 0$ so $0 < \exp(-x) < 1$ so the series holds when $\exp(-x)$ is substituted for $r$.

\begin{equation} \label{eq:06}
\frac{\exp(-x)}{1 - \exp(-x)} = \sum_{k = 1}^{\infty} \left( \exp(-x) \right)^{k}
= \sum_{k = 1}^{\infty} \exp(-kx)
\end{equation}

Mirroring the above progression for the original integral (ignoring the prefactor for the time being)

\begin{align} \label{eq:07}
\int_{x_{g}}^{\infty} \frac{x^{2}}{\exp(x) - 1} dx &= \int_{x_{g}}^{\infty} \frac{x^{2} \exp(-x)}{1 - \exp(-x)} dx \\
 &= \int_{x_{g}}^{\infty} x^{2} \sum_{k = 1}^{\infty} \exp(-kx) dx \\
 &= \sum_{k = 1}^{\infty} \int_{x_{g}}^{\infty} x^{2} \exp(-kx) dx \\
\end{align}

Consider a general term in the series given in Eq. \ref{eq:07}

\begin{equation} \label{eq:08}
\int_{x_{g}}^{\infty} x^{2} \exp(-kx) dx
\end{equation}

Note

\begin{equation} \label{eq:09}
\int_{0}^{\infty} x^{2} \exp(-kx) dx = \int_{x_{g}}^{\infty} x^{2} \exp(-kx) dx + \int_{0}^{x_{g}} x^{2} \exp(-kx) dx
\end{equation}

\noindent so

\begin{equation} \label{eq:10}
\int_{x_{g}}^{\infty} x^{2} \exp(-kx) dx = \int_{0}^{\infty} x^{2} \exp(-kx) dx - \int_{0}^{x_{g}} x^{2} \exp(-kx) dx
\end{equation}

The integral from 0 to $\infty$ can easily be evaluated.

\begin{equation} \label{eq:11}
\int_{0}^{\infty} x^{2} \exp(-kx) dx = \frac{2!}{k^{3}}
\end{equation}

Now consider the integral from 0 to $x_{g}$ and change variable so the integral is evaluated from 0 to 1.

\begin{equation} \label{eq:12}
y \equiv \frac{x}{x_{g}}
\end{equation}

\begin{align} \label{eq:13}
\int_{0}^{x_{g}} x^{2} \exp(-kx) dx &= \int_{0}^{1} (x_{g}y)^{2} \exp(-kx_{g}y) x_{g} dy \\
 &= x_{g}^{3} \int_{0}^{1} y^{2} \exp(-kx_{g}y) dy
\end{align}

Integral 650 of the CRC \cite{} is

\begin{equation} \label{eq:14}
\int_{0}^{1} x^{m} \exp(-ax) dx = \frac{m!}{a^{m+1}} \left( 1 - \exp(-a) \sum_{r = 0}^{m} \frac{a^{r}}{r!} \right)
\end{equation}

So the integral with respect to $y$ from 0 to 1:

\begin{equation} \label{eq:15}
x_{g}^{3} \int_{0}^{1} y^{2} \exp(-kx_{g}y) dy = \frac{2!}{k^{3}} \left( 1 - \exp(-kx_{g}) \sum_{r = 0}^{2} \frac{ (kx_{g})^{r} }{r!} \right)
\end{equation}

Thus,

\begin{equation} \label{eq:16}
\int_{0}^{x_{g}} x^{2} \exp(-kx) dx = \frac{2}{k^{3}} \left( 1 - \exp(-kx_{g}) \left(1 + kx_{g} + \frac{ (kx_{g})^{2} }{2} \right) \right)
\end{equation}

Substitute the results of Eq. \ref{} and \ref{} into Eq. \ref{}

\begin{equation} \label{eq:17}
\int_{x_{g}}^{\infty} x^{2} \exp(-kx) dx = \frac{2}{k^{3}} \exp(-kx_{g}) \left( 1 + kx_{g} + \frac{ (kx_{g})^{2} }{2} \right)
\end{equation}

\noindent which can be substituted back into Eq. \ref{}.

\begin{align} \label{eq:18}
Q(\nu_{g}, T) &= \frac{2\pi (kT)^{3}}{h^{3} c^{2}} \sum_{k = 1}^{\infty} \int_{x_{g}}^{\infty} x^{2} \exp(-kx) dx \\
 &= \frac{2\pi (kT)^{3}}{h^{3} c^{2}} \sum_{k = 1}^{\infty} \frac{2}{k^{3}} \exp(-kx_{g}) \left( 1 + kx_{g} + \frac{ (kx_{g})^{2} }{2} \right) \\
 &= \frac{2\pi (kT)^{3}}{h^{3} c^{2}} \sum_{k = 1}^{\infty} \left( \frac{2}{k^{3}} \exp(-kx_{g}) + \frac{2x_{g}}{k^{2}} \exp(-kx_{g}) + \frac{2x_{g}^{2}}{k} \exp(-kx_{g}) \right) \\
 &= \frac{4\pi (kT)^{3}}{h^{3} c^{2}} \left(\sum_{k = 1}^{\infty} \frac{\exp(-x_{g})^{k}}{k^{3}} + x_{g} \sum_{k = 1}^{\infty} \frac{\exp(-x_{g})^{k}}{k^{2}} + x_{g}^{2} \sum_{k = 1}^{\infty} \frac{\exp(-x_{g})^{k}}{k} \right) \\
\end{align}

The polylogarithm is defined as

\begin{equation} \label{eq:22}
Li_{n}(z) = \sum_{k = 1}^{\infty} \frac{z^{k}}{k^{n}}, |z| < 1
\end{equation}

Recall again that $\exp(-x_{g}) < 1$ since $0 < x_{g} < \infty$. So

\begin{equation} \label{eq:23}
Li_{n} \left(\exp(-x_{g}) \right) = \sum_{k = 1}^{\infty} \frac{\exp(-x_{g})^{k}}{k^{n}}
\end{equation}

Substitute Eq. \ref{} into Eq. \ref{}

\begin{equation} \label{eq:24}
Q(\nu_{g}, T) = \frac{4\pi (kT)^{3}}{h^{3} c^{2}} \left(Li_{3}(\exp(-x_{g})) + x_{g} Li_{2}(\exp(-x_{g})) + x_{g}^{2} Li_{1}(\exp(-x_{g})) \right) 
\end{equation}


% \bibliographystyle{nar}
% \bibliography{}
\end{document}
