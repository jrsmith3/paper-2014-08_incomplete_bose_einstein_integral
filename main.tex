% This file was derived from a template created by Joshua Ryan Smith 
% (joshua.r.smith@gmail.com). The template can be found in the git repo at:
% http://github.com/jrsmith3/latex_template

\documentclass[letterpaper,12pt]{article}

% Document graphics and formatting
% ================================
\usepackage{graphicx}
\graphicspath{{../pics/}}
\usepackage{showkeys}
% \usepackage{endfloat}
% \usepackage{url}
\usepackage{amsmath}
\usepackage[version=3]{mhchem}

% Document metadata
% =================
\newcommand{\Title}{Analytic solution to the incomplete Bose-Einstein integral}
\newcommand{\AuthorName}{Joshua Ryan Smith}
\newcommand{\AuthorEmail}{joshua.smith133.ctr@mail.mil}

\usepackage[pdftex,colorlinks=true,hidelinks]{hyperref}
\hypersetup{
pdftitle={\Title},
pdfauthor={\AuthorName (\AuthorEmail)},
pdfsubject={},
pdfkeywords={},
pdfcreator={pdfTeX}
}

% Define polylogarithm notation for reuse
% =======================================
\newcommand{\Li}{\textrm{Li}}

% Label index
% ===========
% eq:
% --
% 00,01,02,03,04,05,06,07,08,09,
% 10,11,12,13,14,15,16,17,18,19,
% 20,21,22,23,24,25,
% 
% fig
% ---
%
% sec
% ---
%
% tab
% ---

\title{\Title}
\author{\AuthorName}

\begin{document}

\maketitle


\begin{abstract}

\end{abstract}


\section{Introduction}
The incomplete Bose-Einstein integral appears when calculating the flux of above-bandgap photons thermally emitted by an idealized semiconductor. This integral appears when determining the detailed-balance limit of a solar cell \cite{10.1063/1.1736034}, and also when calculating the conduction band population of a material experiencing photon-enhanced thermoelectron emission \cite{10.1038/nmat2814}. Despite the importance of this calculation, everyone seems to leave the details up to the reader. Herein I calculate an analytical solution to the incomplete Bose-Einstein integral in terms of polylogarithm functions.

\section{Calculation}
Shockly and Quiesser \cite{10.1063/1.1736034} give an expression for $Q(\nu_{g}, T)$; the number of photons of frequency greater than $\nu_{g}$ leaving a material per unit area per unit time for blackbody radiation of temperature $T$. This expression is represented in Eq. \ref{eq:00}

\begin{equation} \label{eq:00}
Q(\nu_{g}, T) = \frac{2\pi}{c^{2}} \int_{\nu_{g}}^{\infty} \frac{\nu^{2}}{\exp\left( \frac{h \nu}{kT} \right) - 1} d\nu
\end{equation}

\noindent where $c$ is the speed of light in vacuum, $h$ is Planck's constant, $k$ is Boltzmann's constant, and $\nu$ is the frequency of a photon.

The variable can be changed to simplify the integral as shown in Eq. \ref{eq:01}

\begin{equation} \label{eq:01}
Q(\nu_{g}, T) = \frac{2\pi (kT)^{3}}{h^{3} c^{2}} \int_{x_{g}}^{\infty} \frac{x^{2}}{\exp(x) - 1} dx
\end{equation}

\noindent where $x_{g}$ is defined as

\begin{equation} \label{eq:02}
x_{g}kT = h \nu_{g} = q V_{g}
\end{equation}

\noindent and $q$ is the magnitude of the electron charge and $V_{g}$ is the bandgap of the semiconductor. The exponential term can be factored from the denominator of the integrand to yield Eq. \ref{eq:25}

\begin{equation} \label{eq:25}
Q(\nu_{g}, T) = \frac{2\pi (kT)^{3}}{h^{3} c^{2}} \int_{x_{g}}^{\infty} x^{2} \frac{\exp(-x)}{1 - \exp(-x)} dx
\end{equation}

The geometric series can be rearranged in order to replace the fractional term of the integrand of Eq. \ref{eq:25} with a sum. The geometric series is given in Eq. \ref{eq:03}

\begin{equation} \label{eq:03}
\frac{1}{1-r} = \sum_{n = 0}^{\infty} r^{n}, \qquad |r| < 1
\end{equation}

\noindent Moving the first term of the sum to the left hand side and simplifying yields

\begin{equation} \label{eq:04}
\frac{r}{1-r} = \sum_{n = 1}^{\infty} r^{n}, \qquad |r| < 1
\end{equation}

\noindent Note that $\exp(-x) < 1$ since $0 < x < \infty$; subsituting this exponentiation for $r$ into Eq. \ref{eq:04} yields

\begin{align} \label{eq:06}
\frac{\exp(-x)}{1 - \exp(-x)} &= \sum_{n = 1}^{\infty} \exp(-x)^{n} \nonumber \\
 &= \sum_{n = 1}^{\infty} \exp(-nx)
\end{align}

The sum from Eq. \ref{eq:06} can be substituted into the expression in Eq. \ref{eq:01} to yield

\begin{align} \label{eq:07}
Q(\nu_{g}, T) &= \frac{2\pi (kT)^{3}}{h^{3} c^{2}} \int_{x_{g}}^{\infty} x^{2} \frac{\exp(-x)}{1 - \exp(-x)} dx \nonumber \\
 &= \frac{2\pi (kT)^{3}}{h^{3} c^{2}} \int_{x_{g}}^{\infty} x^{2} \sum_{n = 1}^{\infty} \exp(-nx) dx \nonumber \\
 &= \frac{2\pi (kT)^{3}}{h^{3} c^{2}} \sum_{n = 1}^{\infty} \int_{x_{g}}^{\infty} x^{2} \exp(-nx) dx
\end{align}

Consider a general term in the series given in Eq. \ref{eq:07}

\begin{equation} \label{eq:08}
\int_{x_{g}}^{\infty} x^{2} \exp(-nx) dx
\end{equation}

\noindent Note

\begin{equation} \label{eq:09}
\int_{0}^{\infty} x^{2} \exp(-nx) dx = \int_{x_{g}}^{\infty} x^{2} \exp(-nx) dx + \int_{0}^{x_{g}} x^{2} \exp(-nx) dx
\end{equation}

\noindent so

\begin{equation} \label{eq:10}
\int_{x_{g}}^{\infty} x^{2} \exp(-nx) dx = \int_{0}^{\infty} x^{2} \exp(-nx) dx - \int_{0}^{x_{g}} x^{2} \exp(-nx) dx
\end{equation}

The integral from 0 to $\infty$ can easily be evaluated.

\begin{equation} \label{eq:11}
\int_{0}^{\infty} x^{2} \exp(-nx) dx = \frac{2!}{n^{3}}
\end{equation}

Now consider the integral from 0 to $x_{g}$ and change variable so the integral is evaluated from 0 to 1.

\begin{equation} \label{eq:12}
y \equiv \frac{x}{x_{g}}
\end{equation}

\begin{align} \label{eq:13}
\int_{0}^{x_{g}} x^{2} \exp(-nx) dx &= \int_{0}^{1} (x_{g}y)^{2} \exp(-nx_{g}y) x_{g} dy \nonumber \\
 &= x_{g}^{3} \int_{0}^{1} y^{2} \exp(-nx_{g}y) dy
\end{align}

\noindent Integral 650 of the CRC \cite{} is

\begin{equation} \label{eq:14}
\int_{0}^{1} x^{m} \exp(-ax) dx = \frac{m!}{a^{m+1}} \left( 1 - \exp(-a) \sum_{r = 0}^{m} \frac{a^{r}}{r!} \right)
\end{equation}

\noindent So the integral in Eq. \ref{eq:13} is

\begin{equation} \label{eq:15}
x_{g}^{3} \int_{0}^{1} y^{2} \exp(-nx_{g}y) dy = \frac{2!}{n^{3}} \left( 1 - \exp(-nx_{g}) \sum_{r = 0}^{2} \frac{ (nx_{g})^{r} }{r!} \right)
\end{equation}

\noindent Expressing the left-hand side of Eq. \ref{eq:13} in terms fo the expanded sum of Eq. \ref{eq:15} yields

\begin{equation} \label{eq:16}
\int_{0}^{x_{g}} x^{2} \exp(-nx) dx = \frac{2}{n^{3}} \left( 1 - \exp(-nx_{g}) \left[1 + nx_{g} + \frac{ (nx_{g})^{2} }{2} \right] \right)
\end{equation}

Subtracting Eq. \ref{eq:16} from Eq. \ref{eq:11} as in Eq. \ref{eq:10} yields

\begin{align} \label{eq:17}
\int_{x_{g}}^{\infty} x^{2} \exp(-nx) dx &= \frac{2}{n^{3}} \exp(-nx_{g}) \left( 1 + nx_{g} + \frac{ (nx_{g})^{2} }{2} \right) \nonumber \\
 &= \frac{2}{n^{3}} \exp(-nx_{g}) + \frac{2x_{g}}{n^{2}} \exp(-nx_{g}) + \frac{x_{g}^{2}}{n} \exp(-nx_{g})
\end{align}

\noindent which can be substituted back into Eq. \ref{eq:07} to yield

\begin{align} \label{eq:18}
Q(\nu_{g}, T) &= \frac{2\pi (kT)^{3}}{h^{3} c^{2}} \sum_{n = 1}^{\infty} \int_{x_{g}}^{\infty} x^{2} \exp(-nx) dx \nonumber \\
 &= \frac{2\pi (kT)^{3}}{h^{3} c^{2}} \sum_{n = 1}^{\infty} \left( \frac{2}{n^{3}} \exp(-nx_{g}) + \frac{2x_{g}}{n^{2}} \exp(-nx_{g}) + \frac{2x_{g}^{2}}{n} \exp(-nx_{g}) \right) \nonumber \\
 &= \frac{4\pi (kT)^{3}}{h^{3} c^{2}} \left(\sum_{n = 1}^{\infty} \frac{\exp(-x_{g})^{n}}{n^{3}} + x_{g} \sum_{n = 1}^{\infty} \frac{\exp(-x_{g})^{n}}{n^{2}} + x_{g}^{2} \sum_{n = 1}^{\infty} \frac{\exp(-x_{g})^{n}}{n} \right)
\end{align}

The polylogarithm is defined as

\begin{equation} \label{eq:22}
\Li_{s}(z) = \sum_{n = 1}^{\infty} \frac{z^{n}}{n^{s}}, \qquad |z| < 1
\end{equation}

\noindent Substituting $\exp(-nx_{g})$ as the argument for the polylogarithm yields

\begin{equation} \label{eq:23}
\Li_{s} \left(\exp(-x_{g}) \right) = \sum_{n = 1}^{\infty} \frac{\exp(-x_{g})^{n}}{n^{s}}
\end{equation}

\noindent which can be used to express the summations in Eq. \ref{eq:18}

\begin{equation} \label{eq:24}
Q(\nu_{g}, T) = \frac{4\pi (kT)^{3}}{h^{3} c^{2}} \left(\Li_{3}(\exp(-x_{g})) + x_{g} \Li_{2}(\exp(-x_{g})) + x_{g}^{2} \Li_{1}(\exp(-x_{g})) \right) 
\end{equation}

\noindent Thus, Shockley and Quiesser's photon flux has been expressed in terms of polylogarithms of integer order and real argument.

% \bibliographystyle{nar}
% \bibliography{}
\end{document}
