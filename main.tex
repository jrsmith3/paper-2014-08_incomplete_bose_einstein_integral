% This file was derived from a template created by Joshua Ryan Smith 
% (joshua.r.smith@gmail.com). The template can be found in the git repo at:
% http://github.com/jrsmith3/latex_template

\documentclass[letterpaper,12pt]{article}

% Document graphics and formatting
% ================================
\usepackage{graphicx}
\graphicspath{{img/}{build/}}
\usepackage{showkeys}
% \usepackage{endfloat}
% \usepackage{url}
\usepackage{amsmath}
\usepackage[version=3]{mhchem}

% Document metadata
% =================
\newcommand{\Title}{Analytic solution to the upper incomplete Bose-Einstein integral}
\newcommand{\AuthorName}{Joshua Ryan Smith}
\newcommand{\AuthorEmail}{joshua.smith133.ctr@mail.mil}

\usepackage[pdftex,colorlinks=true,hidelinks]{hyperref}
\hypersetup{
pdftitle={\Title},
pdfauthor={\AuthorName (\AuthorEmail)},
pdfsubject={},
pdfkeywords={},
pdfcreator={pdfTeX}
}

% Define polylogarithm notation for reuse
% =======================================
\newcommand{\Li}{\textrm{Li}}

% Label index
% ===========
% eq:
% --
% 00,01,02,03,04,05,06,07,08,09,
% 10,11,12,13,14,15,16,17,18,19,
% 20,21,22,23,24,25,26,27,28,29,
% 30,31,32,33,34,35,36,37,38,39,
% 40,41,42,43,44,45,46,47,48,49,
% 50,51,52,53,
% 
% fig
% ---
% 00,
%
% sec
% ---
%
% tab
% ---

\title{\Title}
\author{\AuthorName}

\begin{document}

\maketitle


\begin{abstract}
The canonical upper incomplete Bose-Einstein integral is expressed in terms of a finite sum of polylogarithm functions.
\end{abstract}


\section{Introduction}
Computing photon energy and particle fluxes using the incomplete Bose-Einstein integral is central to calculating the efficiency limits of solar cells in the detailed balance limit \cite{10.1063/1.1736034}. This integral also appears in the calculation of the photo-enhanced thermoelectron emission (PETE) from materials \cite{10.1038/nmat2814}. Herein a solution to the upper incomplete Bose-Einstein integral is derived in terms of the polylogarithm function for the case of integer values of the exponent. The polylogarithm has the attractive property of rapid convergence \cite{http://academic.reed.edu/physics/faculty/crandall/papers/Polylog}; a property that will be exploited in this case to rapidly and precisely calculate the numerical value of the incomplete Bose-Einstein integral.


\section{Calculation}
Levy and Honsberg \cite{10.1016/j.sse.2006.06.017} give the incomplete Bose-Einstein integral as 

\begin{equation} \label{eq:26}
F_{m}(E_{A},E_{B},T,\mu) = \frac{2 \pi}{h^{3}c^{2}} \int_{E_{A}}^{E_{B}} E^{m} \frac{1}{\exp \left( \frac{E - \mu}{kT} \right) - 1} dE 
\end{equation}


\noindent for conditions $\mu < E_{A}$ and $0 < E_{A} < E_{B}$; the value of $F_{m}$ is zero when the previous conditions are not met. The quantities $E_{A}$ and $E_{B}$ are the lower and upper limits of the photon emission, respectively, $\mu$ is the equilibrium chemical potential of the radiation, $T$ is the absolute temperature of the solid, $c$ is the speed of light, $h$ is Planck's constant, and $k$ is Boltzmann's constant. The quantity $m$ takes the value of 2 or 3 for photon number flux and photon energy flux, respectively.


% Shockly and Quiesser \cite{10.1063/1.1736034} give an expression for $Q(\nu_{g}, T)$; the number of photons of frequency greater than $\nu_{g}$ leaving a material per unit area per unit time for blackbody radiation of temperature $T$. This expression is represented in Eq. \ref{eq:00}

% \begin{equation} \label{eq:00}
% Q(\nu_{g}, T) = \frac{2\pi}{c^{2}} \int_{\nu_{g}}^{\infty} \frac{\nu^{2}}{\exp\left( \frac{h \nu}{kT} \right) - 1} d\nu
% \end{equation}

% \noindent where $c$ is the speed of light in vacuum, $h$ is Planck's constant, $k$ is Boltzmann's constant, and $\nu$ is the frequency of a photon.

In the following derivation, the incomplete Bose-Einstein integral given in Eq. \ref{eq:26} will be expressed in terms of the polylogarithm function of integer order and real argument. The polylogarithm is defined as

\begin{equation} \label{eq:22}
\Li_{s}(z) = \sum_{n = 1}^{\infty} \frac{z^{n}}{n^{s}}, \qquad |z| < 1
\end{equation}

First, the upper incomplete Bose-Einstein integral will be calculated -- that is, with limits from $E_{A}$ to $\infty$. The resulting expression will be used to calculate the incomplete Bose-Einstein integral given in Eq. \ref{eq:26} from $E_{A}$ to $E_{B}$.


\subsection{Upper-incomplete Bose-Einstein integral}
Consider the upper incomplete Bose-Einstein integral given in Eq. \ref{eq:27} with the variables changed such that

\begin{equation} \label{eq:28}
x \equiv \frac{E}{kT}
\end{equation}

\noindent and

\begin{equation} \label{eq:29}
u \equiv \frac{\mu}{kT}
\end{equation}

\noindent so

\begin{align} \label{eq:27}
F_{m}(E_{A},\infty,T,\mu) &= \frac{2 \pi}{h^{3}c^{2}} \int_{E_{A}}^{\infty} E^{m} \frac{1}{\exp \left( \frac{E - \mu}{kT} \right) - 1} dE \nonumber \\
 &= \frac{2 \pi (kT)^{m+1}}{h^{3}c^{2}} \int_{x_{A}}^{\infty} x^{m} \frac{1}{\exp(x-u) - 1} dx
\end{align}

\noindent where $x_{A} = E_{A}/kT$.

The exponential term can be factored from the denominator of the integrand of Eq. \ref{eq:27} to yield

\begin{equation} \label{eq:30}
F_{m}(\ldots) = \frac{2 \pi (kT)^{m+1}}{h^{3}c^{2}} \int_{x_{A}}^{\infty} x^{m} \frac{\exp(u-x)}{1 - \exp(u-x)} dx
\end{equation}


The geometric series can be rearranged in order to replace the fractional term of the integrand of Eq. \ref{eq:30} with a sum. The geometric series is given in Eq. \ref{eq:03}

\begin{equation} \label{eq:03}
\frac{1}{1-r} = \sum_{n = 0}^{\infty} r^{n}, \qquad |r| < 1
\end{equation}

\noindent Multiplying both sides by $r$ and simplifying yields

\begin{equation} \label{eq:04}
\frac{r}{1-r} = \sum_{n = 1}^{\infty} r^{n}, \qquad |r| < 1
\end{equation}


Note that $\exp(u-x) < 1$ since $u < x$ for all values of $x$ over the entire range of integration as specified in Eq. \ref{eq:27}; substituting this exponentiation for $r$ into Eq. \ref{eq:04} yields

\begin{align} \label{eq:06}
\frac{\exp(u-x)}{1 - \exp(u-x)} &= \sum_{n = 1}^{\infty} \exp(u-x)^{n} \nonumber \\
 &= \sum_{n = 1}^{\infty} \exp(nu) \exp(-nx)
\end{align}


The sum from Eq. \ref{eq:06} can be substituted into the expression in Eq. \ref{eq:30} to yield

\begin{align} \label{eq:07}
F_{m}(\ldots) &= \frac{2\pi (kT)^{m+1}}{h^{3} c^{2}} \int_{x_{A}}^{\infty} x^{m} \frac{\exp(u-x)}{1 - \exp(u-x)} dx \nonumber \\
 &= \frac{2\pi (kT)^{m+1}}{h^{3} c^{2}} \int_{x_{A}}^{\infty} x^{m} \sum_{n = 1}^{\infty} \exp(nu) \exp(-nx) dx \nonumber \\
 &= \frac{2\pi (kT)^{m+1}}{h^{3} c^{2}} \sum_{n = 1}^{\infty} \exp(nu) \int_{x_{A}}^{\infty} x^{m} \exp(-nx) dx
\end{align}

Consider a general term in the series given in Eq. \ref{eq:07}

\begin{equation} \label{eq:08}
\int_{x_{A}}^{\infty} x^{m} \exp(-nx) dx
\end{equation}

\noindent Note

\begin{equation} \label{eq:09}
\int_{0}^{\infty} x^{m} \exp(-nx) dx = \int_{x_{A}}^{\infty} x^{m} \exp(-nx) dx + \int_{0}^{x_{A}} x^{m} \exp(-nx) dx
\end{equation}

\noindent so

\begin{equation} \label{eq:10}
\int_{x_{A}}^{\infty} x^{m} \exp(-nx) dx = \int_{0}^{\infty} x^{m} \exp(-nx) dx - \int_{0}^{x_{A}} x^{m} \exp(-nx) dx
\end{equation}

\noindent The integral from 0 to $\infty$ can easily be evaluated -- \emph{cf}. integral 641 \cite{9780849324796}.

\begin{equation} \label{eq:11}
\int_{0}^{\infty} x^{m} \exp(-nx) dx = \frac{m!}{n^{m+1}}
\end{equation}

Now consider the integral from 0 to $x_{A}$ and change variable so the integral is evaluated from 0 to 1.

\begin{equation} \label{eq:12}
y \equiv \frac{x}{x_{A}}
\end{equation}

\begin{align} \label{eq:13}
\int_{0}^{x_{A}} x^{m} \exp(-nx) dx &= \int_{0}^{1} (x_{A}y)^{m} \exp(-nx_{A}y) x_{A} dy \nonumber \\
 &= x_{A}^{m+1} \int_{0}^{1} y^{m} \exp(-nx_{A}y) dy
\end{align}

\noindent Using integral 650 of \cite{9780849324796}, Eq. \ref{eq:13} becomes

\begin{equation} \label{eq:15}
\int_{0}^{x_{A}} x^{m} \exp(-nx) dx = \frac{m!}{n^{m+1}} \left(1 - \exp(-n x_{A}) \sum_{j = 0}^{m} \frac{(n x_{A})^{j}}{j!} \right) 
\end{equation}

Substituting Eqs. \ref{eq:11} and \ref{eq:15} into Eq. \ref{eq:10} yields

\begin{equation} \label{eq:31}
\int_{x_{A}}^{\infty} x^{m} \exp(-nx) dx = \frac{m!}{n^{m+1}} \exp(-n x_{A}) \sum_{j = 0}^{m} \frac{(n x_{A})^{j}}{j!}
\end{equation}

which can be substituted into Eq. \ref{eq:07} to yield

\begin{align} \label{eq:32}
F_{m}(\ldots) &= \frac{2\pi (kT)^{m+1}}{h^{3} c^{2}} \sum_{n = 1}^{\infty} \exp(nu) \int_{x_{A}}^{\infty} x^{m} \exp(-nx) dx \nonumber \\
 &= \frac{2\pi (kT)^{m+1}}{h^{3} c^{2}} \sum_{n = 1}^{\infty} \exp(nu) \frac{m!}{n^{m+1}} \exp(-n x_{A}) \sum_{j = 0}^{m} \frac{(n x_{A})^{j}}{j!} \nonumber \\
 &= \frac{2\pi m! (kT)^{m+1}}{h^{3} c^{2}} \sum_{n = 1}^{\infty} \exp(nu) \sum_{j = 0}^{m} \frac{x_{A}^{j}}{j!} \frac{\exp(-n x_{A})}{n^{m-j+1}} \nonumber \\
 &= \frac{2\pi m! (kT)^{m+1}}{h^{3} c^{2}} \sum_{n = 1}^{\infty} \exp(nu) \sum_{s = 1}^{m+1} \frac{x_{A}^{m-s+1}}{(m-s+1)!} \frac{\exp(-n x_{A})}{n^{s}} \nonumber \\
 &= \frac{2\pi m! (kT)^{m+1}}{h^{3} c^{2}} \sum_{s = 1}^{m+1} \frac{x_{A}^{m-s+1}}{(m-s+1)!} \sum_{n = 1}^{\infty} \frac{\exp(nu-n x_{A})}{n^{s}} \nonumber \\
 &= \frac{2\pi m! (kT)^{m+1}}{h^{3} c^{2}} \sum_{s = 1}^{m+1} \frac{x_{A}^{m-s+1}}{(m-s+1)!} \sum_{n = 1}^{\infty} \frac{\exp(u-x_{A})^{n}}{n^{s}} \nonumber \\
 &= \frac{2\pi m! (kT)^{m+1}}{h^{3} c^{2}} \sum_{s = 1}^{m+1} \frac{x_{A}^{m-s+1}}{(m-s+1)!} \Li_{s} \left( \exp(u-x_{A}) \right)
\end{align}

\noindent after re-indexing the sum over $j$ to one over $s$ where $s = m - j + 1$. It is appropriate to replace the sum with the polylogarithm function as given in Eq. \ref{eq:22} since, again, $u < x$ and therefore $\exp(u-x) < 1$. Thus, the upper incomplete Bose-Einstein integral has been expressed in terms of a finite sum of polylogarithm functions of integer order and real argument.


\section{Uncertainty analysis}
The precision of the upper-incomplete Bose-Einstein integral given in Eq. \ref{eq:32} is ultimately limited by the precision to which the physical constants are known. The expression of Eq. \ref{eq:32} is the product of two quantities: a coefficient (itself a product of several quantities) and a sum of a series of polylogarithms. The precision to which the series of polylogarithms are calculated is capped by the precision of the coefficient. Computing the value of the polylogarithm series to a precision much higher than the precision of the coefficient is a waste of resources since the subsequent multiplication by the coefficient will wash out any additional precision of the polylogarithm series calculation.

Let

\begin{equation} \label{eq:52}
C = \frac{2 \pi m! (kT)^{m+1}}{h^{3}c^{2}}
\end{equation}

\noindent Analyzing the uncertainty of the coefficient following Taylor \cite{9780935702422} yields

\begin{equation} \label{eq:53}
\frac{\delta C}{|C|} = \left[ (m+1) \left( \frac{\delta k}{k} \right)^{2} + 3 \left( \frac{\delta h}{h} \right)^{2} \right]^{1/2}
\end{equation}

\noindent assuming the quantities 2 and $m$ are known to perfect precision, the value of $\pi$ is known to a precision dramatically exceeding that of the physical constants \cite{http://www.numberworld.org/misc_runs/pi-12t}, and making the generous assumption that $T$ is known to a precision that doesn't limit the precision of $C$. The relative precision of the coefficient $C$ is, at best, on the order of $1 \times 10^{-7}$, based on the most recent CODATA values \cite{10.1103/RevModPhys.84.1527}. Therefore, the series sum of polylogarithms do not need to exceed that value of precision.

The python package \texttt{mpmath} \cite{mpmath} implements Crandall's fast polylogarithm computation algorithm \cite{http://academic.reed.edu/physics/faculty/crandall/papers/Polylog}. This algorithm resolves the polylogarithm at a rate of about 1 precision bit per loop iteration in the worst case would therefore require $1 \times 10^{7}$ iterations to achieve comparable relative precision to the coefficient. Such computuations can easily be executed on commodity computing hardware.


\section{Results}
The incomplete Bose-Einstein integral was implemented in python as a module \cite{10.6084/m9.figshare.1229713} which depends on the \texttt{mpmath} module via the \texttt{sympy} module. The output of the calculation was compared against previously reported values of the efficiencies of a single-junction solar cells. Shockley and Queisser \cite{10.1063/1.1736034} reported the maximum efficiency of a single-junction solar cell is 44\% and has a bandgap of 1.1eV, assuming a solar temperature of 6000K. Using the derivation presented in this paper, an efficiency of 44\% was obtained under identical conditions in good agreement with Shockley and Queisser. In addition, the efficiency vs. bandgap was calculated using the derivation presented in this paper and the result is depicted in Fig. \ref{fig:00}; this figure matches Shockley and Queisser's Fig. 3 \cite{10.1063/1.1736034}.


\begin{figure}
\includegraphics{f00}
\caption{Efficiency vs. bandgap for a single-junction solar cell according to Shockley and Queisser \cite{10.1063/1.1736034}.}
\label{fig:00}
\end{figure}


DeVos \cite{9780198513926} reported a maximum efficiency of 40.8\% for a single-junction solar cell of bandgap 1.15eV illuminated by fully concentrated sunlight of temperature 5762K. An efficiency of 40.5\% was calculated using the derivation described in this paper under identical conditions and assuming the temperature of Earth is 289K.


\section{Colophon}
The model was implemented as a module written in python. Revision 1.0.0 \cite{10.6084/m9.figshare.1229713} of the software implementing the model was used for the calculations in this manuscript along with \texttt{sympy} version 0.7.3, \texttt{numpy} version 1.7.0, \texttt{matplotlib} version 1.3.1, and python version 2.7.7 -- all included with Continuum Analytics's Anaconda distribution version 3.5.5.  The code was executed on a MacBook Air model A1466 running OS X 10.9.5, Darwin kernel version 13F34 with a 2GHz Intel core i7 and 8GB 1600 MHz DDR3 memory.


\section{Acknowledgements}
First, thanks to user Ray Vickson for posting a hint on how to solve the Bose-Einstein integral \cite{https://www.physicsforums.com/threads/a-bose-einstein-integral.751973}, his hint was the germ for this paper. I would like to express extreme gratitude to the staff of the Research Library of the Army Research Laboratory Adelphi Laboratory Center, particularly Nancy Faget, James Agenbroad, Greg Cody, Louise McGovern, and the Stephanie Fields: thank you very much for tracking down all of the books and papers I needed to write this manuscript. Many thanks also to Sasha Wood for tips on how to bound the uncertainty of the polylogarithm function, and thanks to Fredrik Johansson for \texttt{mpmath} and for his help on the uncertainty analysis of the polylogarithm function. Finally, thanks to Randy Tompkins and Lars Pastewka for reading and checking earlier drafts of this manuscript.

Research was sponsored by the Army Research Laboratory and was accomplished under Cooperative Agreement Number W911NF-12-2-0019. The views and conclusions contained in this document are those of the authors and should not be interpreted as representing the official policies, either expressed or implied, of the Army Research Laboratory or the U.S. Government. The U.S. Government is authorized to reproduce and distribute reprints for Government purposes notwithstanding any copyright notation herein.


\bibliographystyle{nar}
\bibliography{bib}
\end{document}
