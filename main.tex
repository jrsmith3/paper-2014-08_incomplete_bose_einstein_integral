% This file was derived from a template created by Joshua Ryan Smith 
% (joshua.r.smith@gmail.com). The template can be found in the git repo at:
% http://github.com/jrsmith3/latex_template

\documentclass[letterpaper,12pt]{article}

% Document graphics and formatting
% ================================
\usepackage{graphicx}
\graphicspath{{../pics/}}
\usepackage{showkeys}
% \usepackage{endfloat}
% \usepackage{url}
\usepackage{amsmath}
\usepackage[version=3]{mhchem}

% Document metadata
% =================
\newcommand{\Title}{Analytic solution to the incomplete Bose-Einstein integral}
\newcommand{\AuthorName}{Joshua Ryan Smith}
\newcommand{\AuthorEmail}{joshua.smith133.ctr@mail.mil}

\usepackage[pdftex,colorlinks=true,hidelinks]{hyperref}
\hypersetup{
pdftitle={\Title},
pdfauthor={\AuthorName (\AuthorEmail)},
pdfsubject={},
pdfkeywords={},
pdfcreator={pdfTeX}
}

% Define polylogarithm notation for reuse
% =======================================
\newcommand{\Li}{\textrm{Li}}

% Label index
% ===========
% eq:
% --
% 00,01,02,03,04,05,06,07,08,09,
% 10,11,12,13,14,15,16,17,18,19,
% 20,21,22,23,24,25,26,27,28,29,
% 30,31,32,33,34,35,36,37,38,39,
% 40,41,
% 
% fig
% ---
%
% sec
% ---
%
% tab
% ---

\title{\Title}
\author{\AuthorName}

\begin{document}

\maketitle


\begin{abstract}
The canonical incomplete Bose-Einstein integral is expressed in terms of a finite sum of polylogarithm functions.
\end{abstract}


\section{Introduction}
Computing photon energy and particle fluxes using the imcomplete Bose-Einstein integral is central to calculating the efficiency of efficiency limits of solar cells in the detailed balance limit \cite{10.1063/1.1736034}. This integral also appears in the calculation of the photo-enhanced thermoelectron emission (PETE) from materials \cite{10.1038/nmat2814}. Herein a solution to the incomplete Bose-Einstein integral is derived in terms of the polylogarithm function for the case of integer values of the exponent. The polylogarithm has the attractive property of rapid convergence via direct summation for small values of the argument \cite{http://academic.reed.edu/physics/faculty/crandall/papers/Polylog}; a property that will be exploited in this case to rapidly and precisely calculate the numerical value of the incomplete Bose-Einstein integral.


\section{Calculation}
Levy and Honsberg \cite{10.1016/j.sse.2006.06.017} give the incomplete Bose-Einstein integral as in Eq. \ref{eq:26}

\begin{equation} \label{eq:26}
F_{m}(E_{A},E_{B},T,\mu) = \frac{2 \pi}{h^{3}c^{2}} \int_{E_{A}}^{E_{B}} E^{m} \frac{1}{\exp \left( \frac{E - \mu}{kT} \right) - 1} dE 
\end{equation}


\noindent for conditions $\mu < E_{A}$ and $0 < E_{A} < E_{B}$; the value of $F_{m}$ is zero when the previous conditions are not met. The quantities $E_{A}$ and $E_{B}$ are the lower and upper limits of the photon emission, respectively, $\mu$ is the equilibrium chemical potential of the radiation, $T$ is the absolute temperature of the solid, $c$ is the speed of light, $h$ is Planck's constant, and $k$ is Boltzmann's constant. The quantity $m$ takes the value of 2 or 3 for photon number flux and photon energy flux, respectively.


% Shockly and Quiesser \cite{10.1063/1.1736034} give an expression for $Q(\nu_{g}, T)$; the number of photons of frequency greater than $\nu_{g}$ leaving a material per unit area per unit time for blackbody radiation of temperature $T$. This expression is represented in Eq. \ref{eq:00}

% \begin{equation} \label{eq:00}
% Q(\nu_{g}, T) = \frac{2\pi}{c^{2}} \int_{\nu_{g}}^{\infty} \frac{\nu^{2}}{\exp\left( \frac{h \nu}{kT} \right) - 1} d\nu
% \end{equation}

% \noindent where $c$ is the speed of light in vacuum, $h$ is Planck's constant, $k$ is Boltzmann's constant, and $\nu$ is the frequency of a photon.

In the following derivation, the incomplete Bose-Einstein integral given in Eq. \ref{eq:26} will be expressed in terms of the polylogarithm function of integer order and real argument. The polylogarithm is defined in Eq. \ref{eq:22}. First, the upper incomplete Bose-Einstein integral will be calculated -- that is, with limits from $E_{A}$ to $\infty$. The resulting expression will be used to calculate the incomplete Bose-Einstein integral given in Eq. \ref{eq:26} from $E_{A}$ to $E_{B}$.

\begin{equation} \label{eq:22}
\Li_{s}(z) = \sum_{n = 1}^{\infty} \frac{z^{n}}{n^{s}}, \qquad |z| < 1
\end{equation}

\subsection{Upper-incomplete Bose-Einstein integral}
Consider the upper incomplete Bose-Einstein integral given in Eq. \ref{eq:27} with the variables changed such that

\begin{equation} \label{eq:28}
x \equiv \frac{E}{kT}
\end{equation}

\noindent and

\begin{equation} \label{eq:29}
u \equiv \frac{\mu}{kT}
\end{equation}

\noindent so

\begin{align} \label{eq:27}
F_{m}(E_{A},\infty,T,\mu) &= \frac{2 \pi}{h^{3}c^{2}} \int_{E_{A}}^{\infty} E^{m} \frac{1}{\exp \left( \frac{E - \mu}{kT} \right) - 1} dE \nonumber \\
 &= \frac{2 \pi (kT)^{m+1}}{h^{3}c^{2}} \int_{x_{A}}^{\infty} x^{m} \frac{1}{\exp(x-u) - 1} dx
\end{align}

\noindent where $x_{A} = E_{A}/kT$.

The exponential term can be factored from the denominator of the integrand of Eq. \ref{eq:27} to yield

\begin{equation} \label{eq:30}
F_{m}(\ldots) = \frac{2 \pi (kT)^{m+1}}{h^{3}c^{2}} \int_{x_{A}}^{\infty} x^{m} \frac{\exp(u-x)}{1 - \exp(u-x)} dx
\end{equation}


The geometric series can be rearranged in order to replace the fractional term of the integrand of Eq. \ref{eq:30} with a sum. The geometric series is given in Eq. \ref{eq:03}

\begin{equation} \label{eq:03}
\frac{1}{1-r} = \sum_{n = 0}^{\infty} r^{n}, \qquad |r| < 1
\end{equation}

\noindent Moving the first term of the sum to the left hand side and simplifying yields

\begin{equation} \label{eq:04}
\frac{r}{1-r} = \sum_{n = 1}^{\infty} r^{n}, \qquad |r| < 1
\end{equation}


Note that $\exp(u-x) < 1$ since $u < x$ for all values of $x$ over the entire range of integration as specified in Eq. \ref{eq:27}; substituting this exponentiation for $r$ into Eq. \ref{eq:04} yields

\begin{align} \label{eq:06}
\frac{\exp(u-x)}{1 - \exp(u-x)} &= \sum_{n = 1}^{\infty} \exp(u-x)^{n} \nonumber \\
 &= \sum_{n = 1}^{\infty} \exp(nu) \exp(-nx)
\end{align}


The sum from Eq. \ref{eq:06} can be substituted into the expression in Eq. \ref{eq:30} to yield

\begin{align} \label{eq:07}
F_{m}(\ldots) &= \frac{2\pi (kT)^{m+1}}{h^{3} c^{2}} \int_{x_{A}}^{\infty} x^{m} \frac{\exp(u-x)}{1 - \exp(u-x)} dx \nonumber \\
 &= \frac{2\pi (kT)^{m+1}}{h^{3} c^{2}} \int_{x_{A}}^{\infty} x^{m} \sum_{n = 1}^{\infty} \exp(nu) \exp(-nx) dx \nonumber \\
 &= \frac{2\pi (kT)^{m+1}}{h^{3} c^{2}} \sum_{n = 1}^{\infty} \exp(nu) \int_{x_{A}}^{\infty} x^{m} \exp(-nx) dx
\end{align}

Consider a general term in the series given in Eq. \ref{eq:07}

\begin{equation} \label{eq:08}
\int_{x_{A}}^{\infty} x^{m} \exp(-nx) dx
\end{equation}

\noindent Note

\begin{equation} \label{eq:09}
\int_{0}^{\infty} x^{m} \exp(-nx) dx = \int_{x_{A}}^{\infty} x^{m} \exp(-nx) dx + \int_{0}^{x_{A}} x^{m} \exp(-nx) dx
\end{equation}

\noindent so

\begin{equation} \label{eq:10}
\int_{x_{A}}^{\infty} x^{m} \exp(-nx) dx = \int_{0}^{\infty} x^{m} \exp(-nx) dx - \int_{0}^{x_{A}} x^{m} \exp(-nx) dx
\end{equation}

\noindent The integral from 0 to $\infty$ can easily be evaluated -- \emph{cf}. integral 641 \cite{9780849324796}.

\begin{equation} \label{eq:11}
\int_{0}^{\infty} x^{m} \exp(-nx) dx = \frac{m!}{n^{m+1}}
\end{equation}

Now consider the integral from 0 to $x_{A}$ and change variable so the integral is evaluated from 0 to 1.

\begin{equation} \label{eq:12}
y \equiv \frac{x}{x_{A}}
\end{equation}

\begin{align} \label{eq:13}
\int_{0}^{x_{A}} x^{m} \exp(-nx) dx &= \int_{0}^{1} (x_{A}y)^{m} \exp(-nx_{A}y) x_{A} dy \nonumber \\
 &= x_{A}^{m+1} \int_{0}^{1} y^{m} \exp(-nx_{A}y) dy
\end{align}

\noindent Using integral 650 of \cite{9780849324796}, Eq. \ref{eq:13} becomes

\begin{equation} \label{eq:15}
\int_{0}^{x_{A}} x^{m} \exp(-nx) dx = \frac{m!}{n^{m+1}} \left(1 - \exp(-n x_{A}) \sum_{j = 0}^{m} \frac{(n x_{A})^{j}}{j!} \right) 
\end{equation}

Substituting Eqs. \ref{eq:11} and \ref{eq:15} into Eq. \ref{eq:10} yields

\begin{equation} \label{eq:31}
\int_{x_{A}}^{\infty} x^{m} \exp(-nx) dx = \frac{m!}{n^{m+1}} \exp(-n x_{A}) \sum_{j = 0}^{m} \frac{(n x_{A})^{j}}{j!}
\end{equation}

which can be substituted into Eq. \ref{eq:07} to yield

\begin{align} \label{eq:32}
F_{m}(\ldots) &= \frac{2\pi (kT)^{m+1}}{h^{3} c^{2}} \sum_{n = 1}^{\infty} \exp(nu) \int_{x_{A}}^{\infty} x^{m} \exp(-nx) dx \nonumber \\
 &= \frac{2\pi (kT)^{m+1}}{h^{3} c^{2}} \sum_{n = 1}^{\infty} \exp(nu) \frac{m!}{n^{m+1}} \exp(-n x_{A}) \sum_{j = 0}^{m} \frac{(n x_{A})^{j}}{j!} \nonumber \\
 &= \frac{2\pi m! (kT)^{m+1}}{h^{3} c^{2}} \sum_{n = 1}^{\infty} \exp(nu) \sum_{j = 0}^{m} \frac{x_{A}^{j}}{j!} \frac{\exp(-n x_{A})}{n^{m-j+1}} \nonumber \\
 &= \frac{2\pi m! (kT)^{m+1}}{h^{3} c^{2}} \sum_{n = 1}^{\infty} \exp(nu) \sum_{s = 1}^{m+1} \frac{x_{A}^{m-s+1}}{(m-s+1)!} \frac{\exp(-n x_{A})}{n^{s}} \nonumber \\
 &= \frac{2\pi m! (kT)^{m+1}}{h^{3} c^{2}} \sum_{s = 1}^{m+1} \frac{x_{A}^{m-s+1}}{(m-s+1)!} \sum_{n = 1}^{\infty} \frac{\exp(nu-n x_{A})}{n^{s}} \nonumber \\
 &= \frac{2\pi m! (kT)^{m+1}}{h^{3} c^{2}} \sum_{s = 1}^{m+1} \frac{x_{A}^{m-s+1}}{(m-s+1)!} \sum_{n = 1}^{\infty} \frac{\exp(u-x_{A})^{n}}{n^{s}} \nonumber \\
 &= \frac{2\pi m! (kT)^{m+1}}{h^{3} c^{2}} \sum_{s = 1}^{m+1} \frac{x_{A}^{m-s+1}}{(m-s+1)!} \Li_{s} \left( \exp(u-x_{A}) \right)
\end{align}

\noindent after re-indexing the sum over $j$ to one over $s$ where $s = m - j + 1$. It is appropriate to replace the sum with the polylogarithm function as given in Eq. \ref{eq:22} since, again, $u < x$ and therefore $\exp(u-x) < 1$. Thus, the upper incomplete Bose-Einstein integral has been expressed in terms of a finite sum of polylogarithm functions of integer order and real argument.

\subsection{Incomplete Bose-Einstein integral}
The incomplete Bose-Einstein integral given in Eq. \ref{eq:26} can be expressed as a difference of two upper-incomplete Bose-Einstein integrals evaluated at different limits.

\begin{align} \label{eq:33}
F_{m}&(E_{A}, E_{B}, T, \mu) = F_{m}(E_{A}, \infty, T, \mu) - F_{m}(E_{B}, \infty, T, \mu) \nonumber \\
 &= \frac{2\pi m! (kT)^{m+1}}{h^{3} c^{2}} \sum_{s = 1}^{m+1} \frac{1}{(m-s+1)!} \left( x_{A}^{m-s+1} \Li_{s} \left( \exp(u-x_{A}) \right) - x_{B}^{m-s+1} \Li_{s} \left( \exp(u-x_{B}) \right)
\right)
\end{align}

\noindent where $x_{B} = E_{B}/kT$.


\section{Uncertainty analysis}
The polylogarithm can be written in terms of an infinite series as in Eq. \ref{eq:22}. An algorithm to approximate the value of the polylogarithm is to simply compute and sum the first $l$ terms of the series and stop when the remainder of the series (the terms from $l+1$ and greater) is accepatably small.

Consider the polylogarithm written as the sum of two terms: the partial sum of the first $l$ terms and the remainder as in Eq. \ref{eq:34}

\begin{equation} \label{eq:34}
\Li_{s}(z) = \Li_{s,l}^{lo}(z) + \Li_{s,l}^{high}(z), \qquad |z| < 1
\end{equation}

\noindent where

\begin{equation} \label{eq:35}
\Li_{s,l}^{lo}(z) = \sum_{n = 1}^{l} \frac{z^{n}}{n^{s}}, \qquad |z| < 1
\end{equation}

\noindent and

\begin{equation} \label{eq:36}
\Li_{s,l}^{high}(z) = \sum_{n = l + 1}^{\infty} \frac{z^{n}}{n^{s}}, \qquad |z| < 1
\end{equation}


Terms should be computed and summed until the ratio

\begin{equation} \label{eq:37}
U = \frac{\Li_{s,l}^{lo}(z)}{\Li_{s,l}^{high}(z)}
\end{equation}

\noindent becomes large. when the ratio in Eq. \ref{eq:37} is large, the remainder, $\Li_{s,l}^{high}(z)$, is small relative to the partial sum $\Li_{s,l}^{lo}(z)$; the magnitude of this ratio determines the significance of the digits given in the decimal representation of $\Li_{s,l}^{lo}(z)$. For example, if the value of $U$ is on the order of $1 \times 10^{10}$, the remainder will only change digits in the $1 \times 10^{11}$ place and below in the decimal representation of the approximation to $\Li_{s}(z)$. Practically speaking, there is no need to compute the value of $\Li_{s}(z)$ to a higher precision than the precision to which the physical constants of Eq. \ref{eq:32} are known.

The value of the uncertainty ratio $U$ can be bounded by bounding both the partial sum and remainder.

The remainder can be bounded above by a (hyper?)-geometric series. Consider the remainder 

\begin{align} \label{eq:38}
\Li_{s,l}^{high}(z) &= \sum_{n = l + 1}^{\infty} \frac{z^{n}}{n^{s}} \\
 &= \frac{z^{l+1}}{(l+1)^{s}} + \frac{z^{l+2}}{(l+2)^{s}} + \frac{z^{l+3}}{(l+3)^{s}} + ...
\end{align}

\noindent This series is bounded above by the corresponding terms of the geometric series

\begin{align} \label{eq:39}
G_{l}^{high}(z) &= \sum_{n = l + 1}^{\infty} z^{n} \\
 &= z^{l+1} + z^{l+2} + z^{l+3} + ...
\end{align}

\noindent This upper bound can be further lowered by dividing through by $(l + 1)^{s}$

\begin{align} \label{eq:40}
\frac{G_{l}^{high}(z)}{(l+1)^{s}} = 
\frac{z^{l+1}}{(l+1)^{s}} + \frac{z^{l+2}}{(l+1)^{s}} + \frac{z^{l+3}}{(l+1)^{s}} + ...
\end{align}

It is clear from the following table that the terms from the remainder of the polylogarithm sum are less than the corresponding terms from the reduced geometric series 

\begin{tabular}{ccc}
$n$ & $\Li_{s,l}^{high}(z)$ & $\frac{G_{l}^{high}(z)}{(l+1)^{s}}$ \\
\hline
$l+1$ & $\frac{z^{l+1}}{(l+1)^{s}}$ & $\frac{z^{l+1}}{(l+1)^{s}}$ \\
$l+2$ & $\frac{z^{l+2}}{(l+2)^{s}}$ & $\frac{z^{l+2}}{(l+1)^{s}}$ \\
$l+3$ & $\frac{z^{l+3}}{(l+3)^{s}}$ & $\frac{z^{l+3}}{(l+1)^{s}}$ \\
... & ... & ... \\
\end{tabular}

\noindent Therefore,

\begin{equation} \label{eq:41}
\Li_{s,l}^{high}(z) < \frac{G_{l}^{high}(z)}{(l+1)^{s}}
\end{equation}


\section{Acknowledgement}
Research was sponsored by the Army Research Laboratory and was accomplished under Cooperative Agreement Number W911NF-12-2-0019. The views and conclusions contained in this document are those of the authors and should not be interpreted as representing the official policies, either expressed or implied, of the Army Research Laboratory or the U.S. Government. The U.S. Government is authorized to reproduce and distribute reprints for Government purposes notwithstanding any copyright notation herein.


\bibliographystyle{nar}
\bibliography{bib}
\end{document}
